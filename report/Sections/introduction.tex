\section{Introduction}
\label{sec:intro}

Being able to reconstruct shapes from partial representations is a useful tool in many medical applications.
For example, one could be interested in the original shape of remaining parts of a bone fracture, which enables building suitable artificial implants.
In this paper we only consider this specific problem of reconstructing femur bones from partial femur shapes.

Knowing typical characteristics of the underlying shape is essential to estimating a high-quality reconstruction.
Such characteristics can be learned from complete data of the corresponding shape family.
We call the result of this learning process a \emph{statistical shape model}.

This paper reports our findings from the project posed in the Future Learn course called ``Statistical Shape Modelling: Computing the Human Anatomy''~\cite{mooc2019statistical}.
The most part of the theory used is also acquired from this course.
