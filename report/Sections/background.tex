\section{Background}
\label{sec:background}

This section provides some fundamental knowledge required to understand our implementations described in Section~\ref{sec:methods}.
However, the theory will only be covered broadly, as it is described in-depth in the according online course~\citep{mooc2019statistical}.

%% --------------------------------------------

\subsection{Gaussian Process}
\label{subsec:gp}

\todoQuestion{Do we want / need that?}

%% --------------------------------------------

\subsection{Statistical Shape Model}
\label{subsec:ssm}

By analyzing a dataset of shapes of the same shape family we can build a statistical shape model (SSM).
We assume that the variations in the available shapes originate in a set of normally distributed features.
For each shape in the dataset, the deformation field to a corresponding reference shape is computed.
From the deformation vectors, we can calculate the mean shape as well as a covariance matrix.
With these two elements \todoRevise{replace word ``elements''} we can fully describe all shapes from the dataset as well as many more similar shapes.



\todoMissing{write background, or maybe leave out because not required according to project intro slides}