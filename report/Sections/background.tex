\section{Background}
\label{sec:background}

This section provides some fundamental knowledge required to understand our implementations described in Section~\ref{sec:methods}.
However, the theory will only be covered broadly, as it is described in-depth in the according online course~\citep{mooc2019statistical}.

%% --------------------------------------------

\subsection{Gaussian Process}
\label{subsec:gp}

According to \citeauthor{seeger2004gaussian}~\cite{seeger2004gaussian}, ``Gaussian processes (GPs) are natural generalisations of multivariate Gaussian random variables to infinite [\dots] index sets.''
A GP is fully defined by a mean function $ \mu : \Omega \rightarrow \mathbb{R}^n $ and a covariance function $k : \Omega \times \Omega \rightarrow \mathbb{R}^{n \times n} $, where $\Omega$ is the domain and $n$ the dimensionality of the data~\citep{mooc2019statistical}.

\todoRevise{Might be incomplete.}

%% --------------------------------------------

\subsection{Statistical Shape Model}
\label{subsec:ssm}

By analyzing a dataset of shapes of the same shape family we can build a statistical shape model (SSM).
We assume that the variations in the available shapes originate in a set of normally distributed features.
For each shape in the dataset, the deformation field to a corresponding reference shape is computed.
From the deformation vectors, we can calculate the mean shape as well as a covariance matrix.
With these, we can fully describe all shapes from the dataset as well as many more similar shapes.

%% --------------------------------------------

\subsection{Distance Metrics}
\label{subsec:metrics}

We estimate the accuracy of our model using two different distance measures.
They value the difference between two meshes.
Both metrics are also used for evaluating the final reconstructions in the SICAS competition~\cite{smir}.

\paragraph{Average Distance}
The average distance computes the average over all distances from every point on one mesh to its closest point on the other mesh:
$$ \frac{1}{|M_1|} \sum_{p \in M_1} \mathit{dist}_{M_2}(p) $$
where $p \in M_1$ denotes a point $p$ on mesh $M_1$ and $\mathit{dist}_{M_2}$ is a function mapping a point to its minimal distance to mesh~$M_2$.

\paragraph{Hausdorff Distance}
Unlike the average distance, this metric computes the difference between two meshes in both directions.
Two shapes are said to be close to each other, if the Hausdorff distance is low.
The distance is calculated as follows~\cite{hausdorff}:
$$ \max \left\{ \begin{array}{ll}
  \sup_{p_1 \in M_1} \inf_{p_2 \in M_2} \mathit{dist}(p_1, p_2), \\
  \sup_{p_2 \in M_2} \inf_{p_2 \in M_2} \mathit{dist}(p_1, p_2) 
\end{array}\right\}. $$
Informally, it describes the maximum over all shortest distances from each point on either mesh to get to the other mesh.
