\section{Conclusion}
\label{sec:conclusion}

We have implemented a working pipeline to get reconstructions of partial femur bones.
The major issue we encountered is most likely based on our prior GP model generated from a kernel function.
It is too flexible at the rough parts of the femur and too inflexible at the more detailed upper and lower part.
We were not able to find a more accurate kernel function within the time range of the project.
Some ideas discussed, which could possibly improve our kernel, are the following:
\begin{itemize}
  \item Kernel scaled individually in each dimension.
  \item Changepoint kernel varying smoothness depending on the position of a point.
\end{itemize}

Furthermore, if we had the possibility to spend more time on computing the interpolation models, the following ideas arise:
\begin{itemize}
  \item Use more or all points on the reference mesh instead of only a subset to compute the registration.
  \item Augment the interpolated model to make it smoother.
   \item Find correspondence using Bayesian theory for fitting instead of ICP.
\end{itemize}

Unlike our model, we have actually learned a lot in this hands-on project: we managed to familiarize ourselves with Scalismo and deepened our understanding of the corresponding online course~\cite{mooc2019statistical}.
Overall we have achieved acceptable results with a clear path for potential improvement.