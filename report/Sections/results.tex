\section{Results}
\label{sec:results}

In this section we present the outcome of our implementation.
\todoRevise{update results with correct values}

%% --------------------------------------------

\subsection{Kernel Model}
\label{subsec:kernmodel}

For constructing a model to register the data, we ended up using was the following: 
$$ k(5, 20) + k(10, 50) + k(100, 200) + ks(200, 500) $$
where $k(s, \sigma)$ is a diagonal kernel of Gaussian kernels with variance $\sigma$ scaled $s$. \todoRevise{I don't think this is still our kernel.}
\autoref{fig:kernel_model} shows its outcome.
The mean of the the kernel model is shown in orange.
Alongside the mean, some samples drawn from its distribution are drawn, and show its variability.

\begin{figure}
	\centering
  \includegraphics[width=\columnwidth]{./Figures/kernel_model_samples}
  \caption{
    The mean of our kernel model shown in orange. 
    It is accompanied by some random samples of the model.}
  \label{fig:kernel_model}
\end{figure}

%% --------------------------------------------

\subsection{Registrations}
\label{subsec:registrresults}
While in the end we are only interested in the distance between our reconstruction and the ground truth, we can also use the average distance and the Hausdorff distance as an indication of how closely we were able to fit our kernel model to the training data. 
In contrast to the previous distances, we are able to calculate these ourselves and hopefully, a better representation of our training data also leads to a better reconstruction of the partial bones. 

We provided mean and standard deviation of these values in \autoref{tbl:registration_distance}.

\begin{table}
\centering
\caption{Distances from fitted model to training data}
\label{tbl:registration_distance}
\begin{tabular}{lrr}
\toprule
\textbf{Bones} &
Average Distance &
Hausdorff Distance \\
\midrule
Mean& 0.63 & 11.05 \\
Standard Deviation& 0.28 & 15.45 \\
\bottomrule
\end{tabular}
\end{table}

In \autoref{fig:registration_fit} you can see an example of the kernel model fitted to the training data.

\begin{figure}
	\centering
  \includegraphics[scale=0.7]{./Figures/registration_fit}
  \caption{Example registration of the model to a bone from the training data}
  \label{fig:registration_fit}
\end{figure}

%% --------------------------------------------

\subsection{Trained Model}
\label{subsec:trainedmodel}
\autoref{fig:trained_model} shows the trained model we generated after fitting our kernel model to the training data.

\begin{figure}
	\centering
  \includegraphics[scale=0.7]{./Figures/trained_model}
  \caption{Mean of the trained model.}
  \label{fig:trained_model}
\end{figure}

%% --------------------------------------------

\subsection{Reconstruction of Partial Bones}
\label{subsec:reconresults}
For the evaluation of the reconstruction of the partial bones, we are interested in closely modelling the actual bone, meaning we want our reconstruction and the ground truth data (not given to us) to be as close as possible. 
This will be tested using both the average distance and the Hausdorff distance between both meshes, both of which should ideally be as small as possible.

\autoref{fig:reconstructed_bone} shows a partial bone next to its reconstruction.

\begin{figure}
	\centering
  \includegraphics[scale=0.7]{./Figures/reconstructed_bone}
  \caption{Example reconstruction of a partial bone.}
  \label{fig:reconstructed_bone}
\end{figure}

All ten images of the partial femur bones that were given alongside our reconstruction can be found in the appendix. 
Their average distance and Hausdorff distance scores can be found in \autoref{tbl:reconstructed_distance}.

\begin{table}
\centering
\caption{Distances from reconstructed bone to ground truth.}
\label{tbl:reconstructed_distance}
\begin{tabular}{lrr}
\toprule
\textbf{Bones} &
Average Distance &
Hausdorff Distance \\
\midrule
Bone 1& 2.44 & 5.40 \\
Bone 2& 3.23 & 10.28 \\
Bone 3& 3.64 & 14.46 \\
Bone 4& 1.92 & 9.46 \\
Bone 5& 5.16 & 40.97 \\
Bone 6& 3.65 & 21.59 \\
Bone 7& 2.65 & 11.91 \\
Bone 8& 3.71 & 12.27 \\
Bone 9& 2.14 & 5.42 \\
Bone 10& 7.46 & 24.89 \\
\midrule
Average & 3.60 & 15.66 \\
\bottomrule
\end{tabular}
\end{table}