\section{Conclusion}
\label{sec:conclusion}

We have described our approach to fit a femur model to 3-D CT images.
In a first step, we used manually clicked landmarks to stretch the model to the large-scale structure of the target shape.
We used MCMC and the Metropolis Hastings algorithm as iterative procedure to do so.
In a second step, we used ASM together with MCMC to get better shape samples.
The results show that the approach yields appropriate results with an average distance of 0.56 millimeters from the ground truth.

The process automates the task of image segmentation and can save time and resources.
Our pipeline could be used as basis for further improving the results.
As we have faced problems with some wrongly clicked landmarks, we see potential in automating that process as well.
Also, finding well suited parameters for all the proposals could be done with a machine learning approach to increase productivity.

The problem of getting stuck in local minima should be avoided when increasing the number of iterations.
However, with our implementation the results got worse when simply increasing the number of iterations.
We have not yet found whether this behavior is the result of a bug in our code or is a phenomena of the random seeds we chose.
It could be crucial to find the source of this behavior to increase the quality of our results.

We have also tried some runs alternating doing some samples of the ASM chain with larger steps and the ASM chain with smaller steps.
However, the measured average distance decreased insignificantly compared to the time spent for the additional rounds.
Thus, the idea was not investigated any further.

\todoRevise{Maybe add some paragraph to make the conclusion more complete?}
