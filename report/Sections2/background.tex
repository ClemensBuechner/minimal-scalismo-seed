\section{Background}
\label{sec:background}

This section introduces the main concepts used to implement our fitting process.
From all concepts learned in the lecture, the final implementation only uses MCMC and ASM.

%% --------------------------------------------

\subsection{Markov Chain Monte Carlo}
\label{subsec:mcmc}

The idea driving MCMC is to sample instances of the shape to approach the underlying true shape.
The algorithm used is called \emph{Metropolis Hastings algorithm}.
It starts each iteration with the last accepted sample and generates a perturbation of it.
The new proposal is then evaluated based on a function estimating its log probability.
The perturbation can be accepted or rejected by the algorithm based on its likelihood ratio when compared with the original sample.
Then, the algorithm will move on to the next iteration.

%% --------------------------------------------

\subsection{Active Shape Models}
\label{subsec:asm}

In our context, an ASM is the combination of a statistical shape model (SSM) and an intensity model.
The latter provides a set of probability distributions representing the intensity variations corresponding to points on the SSM.
We refer to them as profiles.
